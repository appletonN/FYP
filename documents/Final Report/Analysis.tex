
\chapter{Discussion \& Conclusions}

Chapter \ref{Lit:solutions} discussed different solutions that exist to the posed problem of human-centric lighting. How the results discussed in chapter \ref{Sec:physicalOuts} fit into this landscape is discussed below.

\section{A Holistic Solution}

One of the main drawbacks discussed in chapter \ref{Lit:solutions} was the lack of broad-application lights that respect the human both visually and non-visually. Products such as the wake up lamps and Philips HUE exist and can are appropriate for certain times of day, but as shown in chapter \ref{Sec:physicalOuts}, even when using Philips HUE bulbs with specialist software to create a night-friendly light, it causes a 2 and a half hour phase shift of the circadian rhythm.

This device was produced for around £30, plus a further £20 for the display. Considering the economies of scale, this could be significantly reduced. The review of existing solutions highlighted that one of the problems with the current market is the affordability - or lack thereof - of some of the more well developed products. The findings of this project show that a device that is capable of outperforming the high-end Philips HUE smart bulbs can be produced very cheaply. This contradicts the findings of the review, with its conspicuous lack of affordable devices. The exception to this is some of the wake up lights, but their lack of features and specificity of appropriate usage times undermine their effectiveness as a holistic solution.

A normal power cable connected to a 12V transformer (widely available) can be used to power this device, or any other 12V supply - providing the current doesn't exceed the rating of the source (the current drawn by the device depends on how many LEDs are being used). The use of LEDs has significantly reduced the power drawn by the device over traditional lighting methods, and the lifespan will be significantly increased. The modularity of the device further increases the lifespan and accessibility. As mentioned in chapter \ref{sec:Energy}, LEDs can be 100\% efficient if thermal regulation is adequate; the use of large ground planes on the device ensures the best thermal shedding possible which in turn will lead to reduced power consumption for the users. However, the efficiency of the LEDs was not directly tested.

\section{Requirements Analysis}

Of the nine functional requirements laid out in the requirements documentation, five have been successfully achieved, and the other four are still achievable if development of the device were to continue.

The Key Features are much harder to quantitatively assess, which is why they were broken down into the sub-requirement. As each of the high-level requirements can be considered met, it follows that so too can the Key Features.

Many of the implementation options were integrated into the design and even if they are not used, this allows for ``future proofing'' of any modifications and expansions that can be made in the future.

\section{Effect of the Device}

The effects of the device on sleep are briefly discussed in chapter \ref{Sec:CircEffects}. As mentioned there, the device is appropriate for daytime use until a few hours before bed, when an artificial sunset would occur, after which the device can be used up until sleep onset.

But there are more benefits of this than just sleep. The lowering of the \acrshort{cct} will aid evening relaxation of the users, as will the gentle red colour. This goes for the daytime too, when bright, daylight-style LEDs are preferred to induce a comfortable environment.

Alongside the night-time effects, how the device affects mornings should also be considered. The high brightness, especially seen in the Cyanopic and Melanopic ranges, indicate that the device would aid with generating a strong \acrfull{car}. 

This in turn would theoretically correlate to better performance throughout the day, as well as increasing well-being and mood. With this light following from a simulated dawn, the device could certainly help bipolar disorder, \acrshort{sad} and potentially type II diabetes.

Another interesting observation is the similarity of the response of the photoreceptors. The measured Chloropic, Erythropic and Rhodopic lux were all 2390, with the melanopic lux being a close 2420. This means that the experience of the light produced by the device will be very balanced. The Cyanopic lux was the only outlier, but as this photopigment can be more easily damaged by harmful \gls{BlueLight}, this may be a benefit. Further study would need to take place on these effects.


\section{Limitations of Results}

While good insight has been given into the usefulness of the device for sleep and daytime use, the limitations of the methods used must also be considered.

For example, the use of the \acrshort{sam} for this application may not be fully appropriate. This technique is usually used for constituent element identification. However, there is no standard for comparing lighting spectra, hence why this technique was used.

Another limitation was that sleep was not directly measured, but due to the scope of this project, that would not have been feasible. The field of sleep research is vast, and the literature culminates at to the techniques used here to identify the activation of the 5 photopigments and calculating the neurophisiological effects they cause. Similarly, no study on the further consequences of using the light has been conducted, and the effects on mood and disease, although also backed by a lot of research, are speculative.


\section{Other Expected Results}

The further aims of the project, to develop the device as a product and gauge interest, would have been researched using qualitative and quantitative methods. These data would have been used to validate the existing research on mood and lighting, as well as lighting's effects on peoples' impressions of a space.

Unfortunately, this data was not able to be collected due to the pandemic situation. The hypothesis of this area was that the light would give a room a brighter, airy feel when in \acrshort{morning} mode, but calming and relaxing in \acrshort{afternoon} mode. This would have corroborated the findings of the papers outlined in chapter \ref{Sec:mood}.


\section{Evaluation}

\subsection{Project and Process}

Over the course of the project, many difficulties have been faced, with the pandemic causing drastic changes to plans and schedules. From the outset, contingencies and buffers were leveraged to ensure that the main aims could be fulfilled despite unforeseen circumstances.

Due to the careful planning and execution of the project, with a strong focus on de-risking, front-loading work and contingencies, the main aims of the project have been achieved.

From the beginning of the project, there has been a clear flow put in place to ensure that the required tasks get done. By effectively utilising project management techniques and software, the project has remained on course, despite the best efforts of the pandemic to derail it. Solutions to all the major problems that this year has posed were worked through and effective solutions were found. This was possible through communication with specialists, the weekly meetings that have been taking place and through thorough research of the problem and existing or alternative solutions.

Throughout the project, the ability to pro-actively seek solutions to problems that have arisen has improved. Now, at the end of the project, any problems that occur are confidently and swiftly dealt with. 

The interaction with the logbook has also increased. Though at the beginning the logbook was being effectively used, towards the end the updating process of the logbook became a lot more habitual and more effective notes were taken on processes undertaken.

\subsection{UK-SPEC}

The Engineering Council's UK-SPEC is extremely important when considering one's professionalism as an engineer. This project has been designed from the start to reflect as many of the competencies of the UK-SPEC as possible. 

A large portion of the project has been spent designing hardware, through the application of industry-learned knowledge into practical an physical outcomes. This covers competencies \textbf{A} and \textbf{B}.

The project has been independently run and, while collaboration has occurred on the design of the device, the project itself was self-led (competency \textbf{C}). Through meetings and other external communication, the project has been able to reach otherwise inaccessible equipment and knowledge, showing competence in section \textbf{D} of the UK-SPEC.

And finally, a personal commitment has been made to professionalism as an engineer, and the external consequences of undertaking this project. The scrutiny undertaken at the beginning of the project to ensure safe and secure working practices reflects this final competency.

For a full skills-matrix of how each competency has been achieved, see Appendix \ref{}.

The wider social impact of this project has been discussed, but it should be reiterated that this device is fundamentally designed to bring well-being to the built environment, while avoiding the existing problems of distributive justice that currently exist.

\section{Further Research}

Obviously while this project achieved what it set out to do, produce a low cost device that is appropriate for use at any time of day, there is much more that could be done to further this work. Development of the device into a more market-ready product as well as conducting research on the product-market fit and how consumers respond to the product. While this has begun, with product designers being consulted and concept are being produced, there is a lot more that could be done in this area.

Another area of further study could be furthering the technology itself. The current device, while small, portable, easy to use and low power, will not in itself cause a systemic change to how we approach lighting. If the device could be condensed into a much smaller package, a single \acrshort{ic} for example, then designers and developers could use it to create many more solutions that are applicable to an even broader range of problems.

\section{Conclusion}

Through the course and completion of this project, not only has it been shown that a device that a device which is designed around the human responses to light can be produced cheaply, but much personal and professional development has occurred. 

The project has leveraged many facets of professionalism and provided an opportunity to demonstrate effective working practices used to see through the successful completion of all the main aims.

Albeit a shame that the secondary aims could not be undertaken, the pandemic situation has meant that many problems have caused a multitude of changes to the plans of how the project would progress. Despite all of these difficulties, results were still gathered successfully and all the facets of the device that were planned to be quantified successfully were.

The tools used for the management of this project, from the Kanban-style AGILE approach to the use of the logbook have all provided their own benefits and learning points that can be taken into the future as a demonstration of good engineering practice. This is also true of the focus on the  UK-SPEC and a lot of evidence has been gathered for the competencies therein.

On top of all of this, the project has been thoroughly enjoyable throughout, if stressful at times. All of the challenges presented, the work undertaken and research done have all been an opportunity for problem solving, learning, and growth.

